\documentclass[11pt,a4paper,oneside,french,svgnames]{report}
\usepackage[utf8]{inputenc} % force the use of utf8
\usepackage[T1]{fontenc} % font encoding, allows accents
\usepackage[top= 1.5cm,bottom=0.5cm, inner=0.5cm, outer=0.5cm]{geometry} % page formatting
\usepackage[francais]{babel} % translate everything in the desired language: table of contents, etc. 'english' can be replaced with 'francais'
\usepackage{parskip} % disable indent
\usepackage{graphicx} % images management
\usepackage{wrapfig} % floating images
\usepackage{array} % allow arrays
\usepackage{fancyhdr} % headers/footers management (overrides empty, plain and headings)
\usepackage{listings} % code insertion (MUST BE WRITTEN AFTER BABEL)
\usepackage{enumitem} % for /setlist
\usepackage{color,soul} % add some colors and hightlight
\usepackage{xcolor} % more colors
\usepackage[hyphens]{url} % auto break lines in URL
\usepackage[hidelinks,  colorlinks  = true, % no borders, colors enabled
                        anchorcolor = blue,
                        linkcolor   = black, % links in table of contents
                        urlcolor    = blue,
                        citecolor   = blue]{hyperref}

\sethlcolor{cyan} % package soul

%%%%%%%%%%%%%%%%%%%%%%%%%% LISTINGS %%%%%%%%%%%%%%%%%%%%%%%%%%
\definecolor{comment}{rgb}{0.12, 0.38, 0.18 } % adjusted, in Eclipse: {0.25, 0.42, 0.30 } = #3F6A4D
\definecolor{keyword}{rgb}{0.37, 0.08, 0.25}  % #5F1441
\definecolor{string}{rgb}{0.06, 0.10, 0.98} % #101AF9

\lstset{
  columns=flexible, %prevent extra spaces
  rulecolor=\color{black!50},
  backgroundcolor = \color{blue!10},
  numbers=none, % line numbering
  showspaces=false,
  showtabs=false,
  tabsize=4,
  breaklines=true,
  showstringspaces=false,
  breakatwhitespace=false,
  commentstyle=\color{comment},
  keywordstyle=\color{keyword},
  stringstyle=\color{string},
  basicstyle=\ttfamily,
  extendedchars=true,
  emph=[2]{In},
  emphstyle=[2]\color{black!70},
  morecomment=[l][\color{blue}]{Out},
  frame=single,
  frameround=tttt,
  framerule=0.3pt,
  framesep=4pt,
  belowcaptionskip=2.1pt,
  literate={à}{{\`a }}1 {â}{{\^a}}1 %                         letter a
           {À}{{\`A}}1 {Â}{{\^A}}1 %                         letter A
           {ç}{{\c{c}}}1 %                                   letter c
           {Ç}{{\c{C}}}1 %                                   letter C
           {é}{{\'e}}1 {è}{{\`e}}1 {ê}{{\^e}}1 {ë}{{\"e}}1 % letter e
           {É}{{\'E}}1 {È}{{\`E}}1 {Ê}{{\^E}}1 {Ë}{{\"E}}1 % letter E
           {î}{{\^i}}1 {ï}{{\"i}}1 %                         letter i
           {Î}{{\^I}}1 {Ï}{{\"I}}1 %                         letter I
           {ô}{{\^o}}1 %                                     letter o
           {Ô}{{\^O}}1 %                                     letter O
           {œ}{{\oe}}1 %                                     letter oe
           {Œ}{{\OE}}1 %                                     letter OE
           {ù}{{\`u}}1 {û}{{\^u}}1 {ü}{{\"u}}1 %             letter u
           {Ù}{{\`U}}1 {Û}{{\^U}}1 {Ü}{{\"U}}1 %             letter U
  % above is a hack to force UTF8 compatibility (only for french)
}

% Usage : \code{main.cpp}{C++}
\newcommand{\code}[2]{\lstset{
  language=#2,
  title={{\setlength{\fboxsep}{1pt}\fcolorbox{orange}{yellow!20}{\sffamily\scriptsize
              \textcolor{gray!10}{\_}{#1}\textcolor{gray!10}{\_}}}}
  }
}

\title{NF17 - LAPP}
\author{Romain PELLERIN}
\date\today

\makeatletter
\let\thetitle\@title
\makeatother

%\parindent=20pt
\fancypagestyle{plain}{
    % Headers
    \renewcommand{\headrulewidth}{0pt}
    \addtolength{\headsep}{-0.5cm}
    \fancyhead[L]{\color{gray}\thetitle}
    \fancyhead[R]{\color{gray}Dernière MaJ : \today}

    % Footers
    \renewcommand{\footrulewidth}{0pt}
    \fancyfoot{}
}

\setlist[itemize,2]{label={$\bullet$}} % use bullets for nested itemize

\pagestyle{plain} % uses fancy

\begin{document}
\scriptsize

% texte ici si besoin
\code{Export/Import en CSV sous PostgreSQL}{SQL}
\begin{lstlisting}
-- Export
COPY tablename TO 'csv/alldata_export.csv'; -- On peut aussi sélectionner des colonnes avec COPY tablename(col1, col2) ...
COPY (SELECT col1 FROM tablename WHERE ... ) TO 'csv/alldata_export.csv' WITH DELIMITER ',' CSV HEADER; -- Delimiteur CSV, ajoute un header au CSV

-- Import
COPY tablename(col1, col2) FROM 'csv/file.csv' WITH DELIMITER ',' CSV HEADER;

-- Dans un terminal, utiliser \copy
\end{lstlisting}

\code{SQL (postgresql)}{SQL}
\begin{lstlisting}
-- [...] = optionnel

CREATE TYPE TYPE_DEPENSE AS ENUM ('materiel', 'fonctionnement');

DROP TABLE IF EXISTS MATABLE CASCADE; -- CASCADE supprime les vues qui en dépendent, et les contraintes FK dans les autres tables
CREATE TABLE MEMBRE_LABO (                                                        
  matable_id SERIAL PRIMARY KEY, -- SERIAL = INTEGER + crée une séquence automatiquement ; PRIMARY KEY = UNIQUE + NOT NULL
  categorie TEXT NOT NULL CHECK (type = 'I' OR type = 'B'),
  mail_provider FLOAT [CONSTRAINT nom_de_la_contrainte] UNIQUE, -- NULL est autorisé plusieurs fois
  telephone VARCHAR(22),
  date_debut DATE, -- 'YYYY-MM-DD'
  rdv TIMESTAMP REFERENCES RENDEZVOUS, -- 'YYYY-MM-DD HH:MM:SS' -- Référence la primary key de la table RENDEZVOUS
  statut BOOLEAN,
  depense TYPE_DEPENSE NOT NULL
  CHECK (date_debut IS NOT NULL),
  CHECK (mail_provider BETWEEN 1.2 AND 1.5),
  UNIQUE (date_debut, rdv), -- On peut aussi mettre PRIMARY KEY ici, si on ne l'a pas écrit plus haut
  FOREIGN KEY(mail_provider,telephone) REFERENCES USER(mail,telephone) ON DELETE CASCADE -- Les entrées seront supprimées quand le tuple (mail,telephone) n'existe plus dans USER                     
);

CREATE SEQUENCE serial [INCREMENT BY 2] [MINVALUE 1] [MAXVALUE 500] [START 101] [CYCLE]; -- CYCLE = on recommence quand on atteint MAXVALUE
SELECT currval('serial'); -- On peut faire un INSERT INTO xxx VALUES (nextval('serial'));

CREATE VIEW VINGENIEUR AS SELECT id,nom,mail,telephone,validateur,labo,specialite FROM MEMBRE_LABO WHERE type = 'I';

CREATE FUNCTION CHECK_DATE_PROJET_CREATE() RETURNS trigger AS $ret$
  DECLARE
    proposition date;
  BEGIN
    SELECT INTO proposition date_emission FROM PROPOSITION_DE_PROJET prop WHERE prop.appel = NEW.appel;
    IF(NEW.date_debut >= proposition) THEN
      RETURN NEW; -- Les procédure appelées par un trigger BEFORE doivent retourner NEW en cas de succès
    ELSE
      RAISE EXCEPTION 'Date de début antérieure à la proposition de projet !';
  END IF;
  RETURN NULL;
  END;
$ret$ LANGUAGE 'plpgsql'; -- Obligatoire

CREATE TRIGGER TRIGGER_CHECK_DATE_PROJET_CREATE
  BEFORE INSERT ON PROJET
  FOR EACH ROW EXECUTE PROCEDURE CHECK_DATE_PROJET_CREATE();
-------------------------------------------------------------------------------------------------------------------------------------
BEGIN [TRANSACTION]; -- Début transaction
SELECT [DISTINCT] * | colonne [, ...] FROM unetable [, ... ] WHERE condition GROUP BY expression HAVING condition ORDER BY expression [ ASC | DESC ] LIMIT nombre; -- Possibilité de mettre un alias pour colonne (avec le mot-clé AS) et un alias pour table
-- colonne peut contenir COUNT(*), SUM(nb*prix), etc. HAVING condition peut contenir COUNT(id) > 5 par exemple.

INSERT INTO table [(colonne [, ...])] VALUES ({expression | DEFAULT} [, ...]) [RETURNING * | expression [[AS] nomdesortie] [, ...]];

UPDATE table SET {colonne = {expression | DEFAULT}} [, ...] WHERE condition [RETURNING * | expression [[AS] nomdesortie] [, ...]];

DELETE FROM table WHERE condition [RETURNING * | expression [[AS] nomdesortie] [, ...]];
END; -- Fin transaction, COMMIT implicite
\end{lstlisting}

\newpage
\code{HTML de base}{html}
\begin{lstlisting}
<!DOCTYPE html>
<html>
    <head>
        <meta charset="utf-8">
        <link rel="stylesheet" media="all" href="style.css" />
        <title>Site</title>
    </head>
    <body>
      <!-- Contenu ici -->
    </body>
</html>
\end{lstlisting}

\code{PHP et PDO}{PHP}
\begin{lstlisting}
$dbh = new PDO("pgsql:host=domain.tld;port=5432;dbname=database", "username", "password"); // Remplacer domain.tld, database, username et password, port est facultatif, déclenche une PDOException en cas d'erreur lors de la connexion
$dbh->setAttribute(PDO::ATTR_ERRMODE, PDO::ERRMODE_EXCEPTION);
//-----------------------------------------------------------------------------------------------------------------------------------
$res = $this->db->query("SELECT * from MATABLE"); // resultat de la requete contenu dans $res
$rows = $res->fetchAll(PDO::FETCH_CLASS | PDO::FETCH_PROPS_LATE, CLASSE_PHP); // CLASSE_PHP est une classe PHP
// fetchAll est à FETCH_BOTH par défaut (tableau indexé par nom des colonnes et numéro des colonnes, en commençant par 0)
// $rows contient un tableau d'objet de la classe CLASSE_PHP
//-----------------------------------------------------------------------------------------------------------------------------------
$sql = "SELECT * FROM MATABLE WHERE num = :numero AND app = :appel";
$params = array(':numero' =>$numero, ':appel'  =>$appel);
$query = $dbh->prepare($sql);
$res = $query->execute($params);
if ($res===True && query->rowCount()===1) {
  $row = $query->fetch(PDO::FETCH_OBJ); // Crée un objet anonyme ayant les noms de colonnes comme propriétés
  echo $row->NAME;
}
$dbh = NULL; // Coupe la connexion
\end{lstlisting}
\code{PHP et pg\_connect}{PHP}
\begin{lstlisting}
$conn = pg_connect("host=domain.tld port=5432 dbname=database user=username password=password") or die("Connexion impossible");
if (!$dbh) {
  echo "Erreur";
  exit;
}
$result = pg_query($conn, "SELECT auteur, email FROM auteurs"); // On peut mettre plusieurs statements séparés par des ";"
if (!$result) {
  echo "Erreur";
  exit;
}
$nombre_de_resultats = pg_num_rows($result);
while ($row = pg_fetch_row($result)) {
  echo "Auteur: $row[0]  E-mail: $row[1]";
  echo "<br />\n";
}
//-----------------------------------------------------------------------------------------------------------------------------------
$arr = pg_fetch_array($result, 0, PGSQL_BOTH);
echo $arr[0] . " <- Row 1 Author\n";
pg_close($conn);
\end{lstlisting}

\end{document}
